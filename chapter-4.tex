\chapter{Non-Linearity}

\section{Global Non-Linearity}

We determined the global non-linearity by fitting a quadratic function to the mean value of images in a ramp, after performing first-order reference correction. We used the specific model:
\begin{eqnarray}
    \bar n = a + b i + c i^2,
\end{eqnarray}
in which $\bar n$ is the mean value and $i$ is the image sequence number. The corresponding nominal linear model is just $a + bi$.

The first two images in a ramp seem to show anomalous values (perhaps a timing anomaly) and so were not included in the fit. We fitted up to images with 45,000 DN, which we empirically determined is close to point beyond which a quadratic fit is no longer adequate.

The global fit is shown in Figure~\ref{figure:linearity-data} and the residuals (the ratio of the data to the fitted model) in Figure~\ref{figure:linearity-residual}. The fitted coefficients are:
\begin{eqnarray}
    a &=& 1764\\
    b &=& 2229\\
    c &=& -11.41
\end{eqnarray}
We can see that the detector is well modelled by a quadratic response below 45,000 DN, with errors of much less than 1\%, but this model becomes inadequate at about 50,000 DN as the detector saturates. Our  limit of 45,000 DN corresponds to about 100,000~$e$.

\begin{figure}
    \centering

    \includegraphics[width=0.9\linewidth]{chapter-4/linearity-data.png}
    \caption{The fit for the global non-linearity model. The limits of the fit are shown with horizontal dashed lines. Both the quadratic fit and the nominal linear model are shown.}
    \label{figure:linearity-data}

    \includegraphics[width=0.9\linewidth]{chapter-4/linearity-residual.png}
    \caption{The residuals (the ratio of the data to the fitted model) for the global non-linearity model. The anomalies in the first two images are obvious.}
    \label{figure:linearity-residual}

\end{figure}

We can quantify the non-linearity by the ratio $f$ between the non-linear model and the nominal linear model:
\begin{eqnarray}
    f &=& \frac{a + b i + c i^2}{a + b i}
\end{eqnarray}
This is shown in Figure~\ref{figure:linearity-correction}. We see that the non-linearity is approximately 10\% at 45,000 DN.

\begin{figure}
    \centering

    \includegraphics[width=0.9\linewidth]{chapter-4/linearity-correction.png}
    \caption{The fit for the global non-linearity ratio $f$.}
    \label{figure:linearity-correction}

\end{figure}

\section{Channel-to-Channel Variation}

We might expect each channel to have a slightly different non-linearity. We can quantify this by fitting a model to the pixels for each channel and then plotting $c/b$. We see that this varies across the detector from about $-0.004$ to $-0.006$. This suggests that it may be worth considering correcting the non-linearity on a channel-by-channel basis.

\begin{figure}
    \centering

    \includegraphics[width=0.9\linewidth]{chapter-4/linearity-channel.png}
    \caption{The value of $c/b$ for each channel.}
    \label{figure:linearity-channel}

\end{figure}

\section{Comparison to Values Measured at IAC}

The comparison to the non-linearity measured at IAC is not trivial. The figure on page 37 of \cite{rodriguez-2025} shows larger deviations, but this appears to be the result of fitting a model well beyond the limit 45,000 DN that we consider appropriate.
