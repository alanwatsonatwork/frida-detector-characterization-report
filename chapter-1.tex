\chapter{Introduction}

This document describes the process and results of the characterization of the FRIDA H2RG detector in the laboratory at UNAM in México City in fall 2025.

\section{Data-Taking Environment}

All images were taken with the laboratory data acquisition system provided by IAC. All data were taken with a gain mode of 10.

\section{Dark Configuration}

Some of our tests require that the detector not be illuminated. TODO: description of configuration.

\section{Illumination Environment}

Some of our tests require that the detector be illuminated. For these, we used the FRIDA calibration unit. This is an external system that can place lamps in front for the cryostat window and feed the optical path with an approximately XX beam from either a stabilized tungsten lamp and an arc lamp are available. We used the tungsten lamp.

We operated the instrument in imaging mode. The calibration unit is designed mainly to calibrate the IFS mode and so does not uniformly illuminate the field of view in imaging mode. To mitigate this, we used it with the fine camera. However, as the focal plane mask for the fine camera slightly vignettes the edges of the field, we used the larger focal plane mask for the medium camera.

With the inscribed pupil stop, the NB2.48 filter, and the ND2 filter, and with the lamp supplied with 1.07 A, the count rate was about XX DN/s.
