\chapter{Summary}

We have characterized the FRIDA science detector in the FRIDA cryostat at UNAM. We find:
\begin{itemize}
    \item First-order channel correction works well.
    \item However, the images have residual horizontal striping at a low level. We have not attempted to correct this, but have suggested a possible solution if this is a problem in science data.
    \item The read noise is about 21 $e$ and the gain is about 2.24 $e/\mathrm{DM}$. These values are in fairly good agreement with the values measured at IAC.
    \item The non-linearity appears to be correctable to better than 1\% up to about 45,000 DN or 100,000 $e$.
    \item The first two images in ramps show anomalous values.
\end{itemize}
We conclude that the detector appears to be perfectly adequate for science observations.